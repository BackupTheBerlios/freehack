\section{versuch2/animation.h-Dateireferenz}
\label{animation_8h}\index{versuch2/animation.h@{versuch2/animation.h}}
In dieser Datei wird die Klasse {\bf CAnimation}{\rm (S.\,\pageref{classCAnimation})} definiert. 

{\tt \#include $<$SDL/SDL.h$>$}\par
{\tt \#include $<$string$>$}\par
{\tt \#include \char`\"{}animationimage.h\char`\"{}}\par
{\tt \#include \char`\"{}surfacechild.h\char`\"{}}\par
\subsection*{Klassen}
\begin{CompactItemize}
\item 
class {\bf CAnimation}
\begin{CompactList}\small\item\em CAnimation ist eine Klasse um Animationen zu laden und zu zeichnen. \item\end{CompactList}\end{CompactItemize}


\subsection{Ausf\"{u}hrliche Beschreibung}
In dieser Datei wird die Klasse {\bf CAnimation}{\rm (S.\,\pageref{classCAnimation})} definiert. 

{\bf CAnimation}{\rm (S.\,\pageref{classCAnimation})} kann dazu genutzt werden, Animationen zu laden und auf zu zeichnen.

\begin{Desc}
\item[Autor:]Bodo Akdeniz\end{Desc}
