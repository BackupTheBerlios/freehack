\section{CAnimation\-Image Klassenreferenz}
\label{classCAnimationImage}\index{CAnimationImage@{CAnimationImage}}
Die Klasse dient als Frame von {\bf CAnimation}{\rm (S.\,\pageref{classCAnimation})}.  


{\tt \#include $<$animationimage.h$>$}

Klassendiagramm f\"{u}r CAnimation\-Image::\begin{figure}[H]
\begin{center}
\leavevmode
\includegraphics[height=3cm]{classCAnimationImage}
\end{center}
\end{figure}
\subsection*{\"{O}ffentliche Methoden}
\begin{CompactItemize}
\item 
{\bf CAnimation\-Image} (SDL\_\-Surface $\ast$dest)
\begin{CompactList}\small\item\em Constructor. \item\end{CompactList}\item 
virtual {\bf $\sim$CAnimation\-Image} ()
\begin{CompactList}\small\item\em Destructor. \item\end{CompactList}\item 
virtual void {\bf Set\-Delay\-Time} (int delaytime)
\begin{CompactList}\small\item\em Setzt {\em Delay\-Time\/} auf {\em delaytime\/}. \item\end{CompactList}\item 
virtual int {\bf Get\-Delay\-Time} ()
\begin{CompactList}\small\item\em Gibt {\em Delay\-Time\/} zur\"{u}ck. \item\end{CompactList}\item 
virtual void {\bf Set\-Timestamp} (int n)
\begin{CompactList}\small\item\em Setzt {\em timestamp\/} auf {\em n\/}. \item\end{CompactList}\item 
virtual int {\bf Get\-Timestamp} ()
\begin{CompactList}\small\item\em Gibt {\em timestamp\/} zur\"{u}ck. \item\end{CompactList}\end{CompactItemize}
\subsection*{\"{O}ffentliche Attribute}
\begin{CompactItemize}
\item 
int {\bf Delay\-Time}
\begin{CompactList}\small\item\em {\em Delay\-Time\/} legt eine Zeit in Milisekunden fest, die gewartet werden soll, bis der n\"{a}chste Frame von {\bf CAnimation}{\rm (S.\,\pageref{classCAnimation})} angezeigt wird. \item\end{CompactList}\item 
int {\bf timestamp}
\begin{CompactList}\small\item\em Die Anzahl Milisekunden, die vom Beginn der Animation ({\bf CAnimation}{\rm (S.\,\pageref{classCAnimation})}) an verstreichen, bis dieser Frame gezeichnet wird. \item\end{CompactList}\end{CompactItemize}


\subsection{Ausf\"{u}hrliche Beschreibung}
Die Klasse dient als Frame von {\bf CAnimation}{\rm (S.\,\pageref{classCAnimation})}. 

Sie erweitert {\bf CImage}{\rm (S.\,\pageref{classCImage})} um {\em Delay\-Time\/}, um sie ihre Eignung f\"{u}r Animationen zu bessern.

\begin{Desc}
\item[Autor:]Bodo Akdeniz \end{Desc}
\begin{Desc}
\item[Datum:]01.04.05 \end{Desc}
\begin{Desc}
\item[Version:]0.2 \end{Desc}




\subsection{Beschreibung der Konstruktoren und Destruktoren}
\index{CAnimationImage@{CAnimation\-Image}!CAnimationImage@{CAnimationImage}}
\index{CAnimationImage@{CAnimationImage}!CAnimationImage@{CAnimation\-Image}}
\subsubsection{\setlength{\rightskip}{0pt plus 5cm}CAnimation\-Image::CAnimation\-Image (SDL\_\-Surface $\ast$ {\em dest})}\label{classCAnimationImage_a0}


Constructor. 

Siehe {\bf CImage::CImage()}{\rm (S.\,\pageref{classCImage_a0})}\index{CAnimationImage@{CAnimation\-Image}!~CAnimationImage@{$\sim$CAnimationImage}}
\index{~CAnimationImage@{$\sim$CAnimationImage}!CAnimationImage@{CAnimation\-Image}}
\subsubsection{\setlength{\rightskip}{0pt plus 5cm}virtual CAnimation\-Image::$\sim${\bf CAnimation\-Image} ()\hspace{0.3cm}{\tt  [virtual]}}\label{classCAnimationImage_a1}


Destructor. 



\subsection{Dokumentation der Elementfunktionen}
\index{CAnimationImage@{CAnimation\-Image}!GetDelayTime@{GetDelayTime}}
\index{GetDelayTime@{GetDelayTime}!CAnimationImage@{CAnimation\-Image}}
\subsubsection{\setlength{\rightskip}{0pt plus 5cm}virtual int CAnimation\-Image::Get\-Delay\-Time ()\hspace{0.3cm}{\tt  [virtual]}}\label{classCAnimationImage_a3}


Gibt {\em Delay\-Time\/} zur\"{u}ck. 

\begin{Desc}
\item[R\"{u}ckgabe:]{\em Delay\-Time\/}.\end{Desc}
\index{CAnimationImage@{CAnimation\-Image}!GetTimestamp@{GetTimestamp}}
\index{GetTimestamp@{GetTimestamp}!CAnimationImage@{CAnimation\-Image}}
\subsubsection{\setlength{\rightskip}{0pt plus 5cm}virtual int CAnimation\-Image::Get\-Timestamp ()\hspace{0.3cm}{\tt  [virtual]}}\label{classCAnimationImage_a5}


Gibt {\em timestamp\/} zur\"{u}ck. 

\begin{Desc}
\item[R\"{u}ckgabe:]Die Funktion gibt den Wert von {\em timestamp\/} zur\"{u}ck. \end{Desc}
\begin{Desc}
\item[Siehe auch:]{\bf Set\-Timestamp()}{\rm (S.\,\pageref{classCAnimationImage_a4})}\end{Desc}
\index{CAnimationImage@{CAnimation\-Image}!SetDelayTime@{SetDelayTime}}
\index{SetDelayTime@{SetDelayTime}!CAnimationImage@{CAnimation\-Image}}
\subsubsection{\setlength{\rightskip}{0pt plus 5cm}virtual void CAnimation\-Image::Set\-Delay\-Time (int {\em delaytime})\hspace{0.3cm}{\tt  [virtual]}}\label{classCAnimationImage_a2}


Setzt {\em Delay\-Time\/} auf {\em delaytime\/}. 

\begin{Desc}
\item[Parameter:]
\begin{description}
\item[{\em delaytime}]ist die Zeit (in Milisekunden) auf die {\em Delay\-Time\/} gesetzt werden soll.\end{description}
\end{Desc}
\index{CAnimationImage@{CAnimation\-Image}!SetTimestamp@{SetTimestamp}}
\index{SetTimestamp@{SetTimestamp}!CAnimationImage@{CAnimation\-Image}}
\subsubsection{\setlength{\rightskip}{0pt plus 5cm}virtual void CAnimation\-Image::Set\-Timestamp (int {\em n})\hspace{0.3cm}{\tt  [virtual]}}\label{classCAnimationImage_a4}


Setzt {\em timestamp\/} auf {\em n\/}. 

Die Funktion setzt {\em timestamp\/} auf {\em n\/}. Diese Funktion muss nie von Hand aufgerufen werden. Sie wird von {\bf CAnimation}{\rm (S.\,\pageref{classCAnimation})} ben\"{o}tigt und aufgerufen. \begin{Desc}
\item[Parameter:]
\begin{description}
\item[{\em n}]ist der Wert auf den {\em timestamp\/} gesetzt wird. \end{description}
\end{Desc}
\begin{Desc}
\item[Siehe auch:]{\bf Get\-Timestamp()}{\rm (S.\,\pageref{classCAnimationImage_a5})}\end{Desc}


\subsection{Dokumentation der Datenelemente}
\index{CAnimationImage@{CAnimation\-Image}!DelayTime@{DelayTime}}
\index{DelayTime@{DelayTime}!CAnimationImage@{CAnimation\-Image}}
\subsubsection{\setlength{\rightskip}{0pt plus 5cm}int {\bf CAnimation\-Image::Delay\-Time}}\label{classCAnimationImage_o0}


{\em Delay\-Time\/} legt eine Zeit in Milisekunden fest, die gewartet werden soll, bis der n\"{a}chste Frame von {\bf CAnimation}{\rm (S.\,\pageref{classCAnimation})} angezeigt wird. 

\index{CAnimationImage@{CAnimation\-Image}!timestamp@{timestamp}}
\index{timestamp@{timestamp}!CAnimationImage@{CAnimation\-Image}}
\subsubsection{\setlength{\rightskip}{0pt plus 5cm}int {\bf CAnimation\-Image::timestamp}}\label{classCAnimationImage_o1}


Die Anzahl Milisekunden, die vom Beginn der Animation ({\bf CAnimation}{\rm (S.\,\pageref{classCAnimation})}) an verstreichen, bis dieser Frame gezeichnet wird. 



Die Dokumentation f\"{u}r diese Klasse wurde erzeugt aufgrund der Datei:\begin{CompactItemize}
\item 
versuch2/{\bf animationimage.h}\end{CompactItemize}
