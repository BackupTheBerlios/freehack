\section{Die Benutzung von CList}\label{clist_usage}
{\bf CList}{\rm (S.\,\pageref{da/d2a/classCList})} ist von {\bf std::vector}{\rm (S.\,\pageref{dd/d4e/classstd_1_1vector})} abgeleitet und hat deshalb auch genau die selben Eigenschaften und Funktionen. Man kann {\bf CList}{\rm (S.\,\pageref{da/d2a/classCList})} genauso wie {\bf std::vector}{\rm (S.\,\pageref{dd/d4e/classstd_1_1vector})} benutzen, allerdings wurden auch neue Funktionen hinzugef\"{u}gt, die allerdings nur dazu dienen, den Aufruf der Vektorklasse passender zu den restlichen Funktionen zu gestalten.\subsection{Eine einfache Liste}\label{clist_usage_einfacheliste}
Eine Neue Liste legt man mit 

\footnotesize\begin{verbatim} CList<type> liste;
\end{verbatim}
\normalsize
an. {\tt type} muss durch den Variablentyp ersetzt werden, von dem die Elemente der Liste sein sollen. Dann kann man mit 

\footnotesize\begin{verbatim} liste.Add(object);
\end{verbatim}
\normalsize
Objekte (vom Typ {\tt type}) in die List aufnehmen.\par
 Der Aufruf der einzelnen Listenelemente erfolgt wie bei einem Array mit 

\footnotesize\begin{verbatim} liste[elementnummer];
\end{verbatim}
\normalsize
{\tt elementnummer} geht dabei von 0 bis {\tt liste.Count()-1}. 

\footnotesize\begin{verbatim} liste.Count();
\end{verbatim}
\normalsize
gibt die Anzahl der Elemente, die in der Liste gespeichert sind zur\"{u}ck. Die Nummer des gr\"{o}\ss{}en Elements ist folglich {\tt liste.Count()-1}, da das erste Element die Nummer {\em 0\/} hat. \par
 \par
 Ein Beispiel befindet sich in der Datei {\bf CList1::cc}{\rm (S.\,\pageref{d8/da3/CList1_8cc-example})}\subsection{Eine Liste aus Zeigern}\label{clist_usage_zeigerliste}
Eine Liste kann nat\"{u}rlich auch aus Zeigern bestehen. In diesem Fall kann man der Liste auch {\tt NULL} hinzuf\"{u}gen. D.h. das Objekt, das man der Liste hinzuf\"{u}gen will muss noch nicht existieren und kann erst im nachhinein erstellt werden. Au\ss{}erdem muss bei komplexen Objekten nicht das ganze Objekt kopiert werden, da in der Liste ja nur ein Zeiger auf das Objekt gespeichert wird. 

\footnotesize\begin{verbatim} myobject object1, *object2;
 CList<myobject*> liste;
 // Nullzeiger hinzuf�gen
 liste.Add(NULL);
 liste[liste.Count()-1] = new myobject;
 //Zeiger auf Objekt hinzuf�gen
 liste.Add(&object1);
 liste.Add(object2);
\end{verbatim}
\normalsize
Ein Beispiel dazu befindet sich in der Datei {\bf CList2::cc}{\rm (S.\,\pageref{dd/d12/CList2_8cc-example})} 