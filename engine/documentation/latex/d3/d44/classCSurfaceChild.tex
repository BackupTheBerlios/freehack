\section{CSurface\-Child Klassenreferenz}
\label{d3/d44/classCSurfaceChild}\index{CSurfaceChild@{CSurfaceChild}}
CSurface\-Child ist ein, in einem SDL\_\-Surface, positionnierbares Object.  


{\tt \#include $<$surfacechild.h$>$}

Klassendiagramm f\"{u}r CSurface\-Child::\begin{figure}[H]
\begin{center}
\leavevmode
\includegraphics[height=2cm]{d3/d44/classCSurfaceChild}
\end{center}
\end{figure}
\subsection*{\"{O}ffentliche Methoden}
\begin{CompactItemize}
\item 
{\bf CSurface\-Child} (SDL\_\-Surface $\ast$surface)
\begin{CompactList}\small\item\em Constuctor. \item\end{CompactList}\item 
virtual {\bf $\sim$CSurface\-Child} ()
\begin{CompactList}\small\item\em Destructor. \item\end{CompactList}\item 
virtual void {\bf Set\-Destination\-Surface} (SDL\_\-Surface $\ast$dest)
\begin{CompactList}\small\item\em Setzt {\em destination\/} auf {\em dest\/}. \item\end{CompactList}\item 
virtual void {\bf Set\-Position} (int newx, int newy)
\begin{CompactList}\small\item\em Gibt die Position an, an der das Bild gespeichert werden soll. \item\end{CompactList}\item 
virtual int {\bf Get\-X} ()
\begin{CompactList}\small\item\em Gibt {\em x\/} zur\"{u}ck. \item\end{CompactList}\item 
virtual int {\bf Get\-Y} ()
\begin{CompactList}\small\item\em Gibt {\em y\/} zur\"{u}ck. \item\end{CompactList}\item 
virtual SDL\_\-Surface $\ast$ {\bf Get\-Destination\-Surface} ()
\begin{CompactList}\small\item\em Gibt {\em destination\/} zur\"{u}ck. \item\end{CompactList}\end{CompactItemize}
\subsection*{\"{O}ffentliche Attribute}
\begin{CompactItemize}
\item 
SDL\_\-Surface $\ast$ {\bf destination}
\begin{CompactList}\small\item\em Das SDL\_\-Surface in dem die Ausgabe stattfinden bzw. in dem gezeichnet wird. \item\end{CompactList}\item 
int {\bf x}
\begin{CompactList}\small\item\em Die x-Position des Objekts relativ zu {\em destination\/}. \item\end{CompactList}\item 
int {\bf y}
\begin{CompactList}\small\item\em Die y-Position des Objekts relativ zu {\em destination\/}. \item\end{CompactList}\end{CompactItemize}


\subsection{Ausf\"{u}hrliche Beschreibung}
CSurface\-Child ist ein, in einem SDL\_\-Surface, positionnierbares Object. 

Von dieser Klasse werden alle anderen Objekte abgeleitet, die in ein SDL\_\-Surface (auch der Bildschirm ist in SDL ein SDL\_\-Surface) gezeichnet werden und eine bestimmte Position haben. Das d\"{u}rfte auf alle Objekte zutreffen, die irgendwann auf dem Bildschirm zu sehen sind.

\begin{Desc}
\item[Autor:]Bodo Akdeniz \end{Desc}
\begin{Desc}
\item[Datum:]01.04.05 \end{Desc}
\begin{Desc}
\item[Version:]0.1 \end{Desc}




Definiert in Zeile 24 der Datei surfacechild.h.

\subsection{Beschreibung der Konstruktoren und Destruktoren}
\index{CSurfaceChild@{CSurface\-Child}!CSurfaceChild@{CSurfaceChild}}
\index{CSurfaceChild@{CSurfaceChild}!CSurfaceChild@{CSurface\-Child}}
\subsubsection{\setlength{\rightskip}{0pt plus 5cm}CSurface\-Child::CSurface\-Child (SDL\_\-Surface $\ast$ {\em surface})}\label{d3/d44/classCSurfaceChild_a0}


Constuctor. 

Initiert das Objekt und setzt {\em destination\/} auf {\em surface\/}. \begin{Desc}
\item[Parameter:]
\begin{description}
\item[{\em surface}]ist ein Zeiger auf das SDL\_\-Surface auf dem dann gezeichnet werden soll.\end{description}
\end{Desc}


Definiert in Zeile 11 der Datei surfacechild.cc.

Benutzt Set\-Destination\-Surface().\index{CSurfaceChild@{CSurface\-Child}!~CSurfaceChild@{$\sim$CSurfaceChild}}
\index{~CSurfaceChild@{$\sim$CSurfaceChild}!CSurfaceChild@{CSurface\-Child}}
\subsubsection{\setlength{\rightskip}{0pt plus 5cm}CSurface\-Child::$\sim${\bf CSurface\-Child} ()\hspace{0.3cm}{\tt  [virtual]}}\label{d3/d44/classCSurfaceChild_a1}


Destructor. 



Definiert in Zeile 16 der Datei surfacechild.cc.

\subsection{Dokumentation der Elementfunktionen}
\index{CSurfaceChild@{CSurface\-Child}!GetDestinationSurface@{GetDestinationSurface}}
\index{GetDestinationSurface@{GetDestinationSurface}!CSurfaceChild@{CSurface\-Child}}
\subsubsection{\setlength{\rightskip}{0pt plus 5cm}SDL\_\-Surface $\ast$ CSurface\-Child::Get\-Destination\-Surface ()\hspace{0.3cm}{\tt  [virtual]}}\label{d3/d44/classCSurfaceChild_a6}


Gibt {\em destination\/} zur\"{u}ck. 

\begin{Desc}
\item[R\"{u}ckgabe:]Gibt einen Zeiger auf das SDL\_\-Surface zur\"{u}ck auf dem gezeichnet werden soll.\end{Desc}


Definiert in Zeile 41 der Datei surfacechild.cc.

Wird benutzt von CAnimation::Draw().\index{CSurfaceChild@{CSurface\-Child}!GetX@{GetX}}
\index{GetX@{GetX}!CSurfaceChild@{CSurface\-Child}}
\subsubsection{\setlength{\rightskip}{0pt plus 5cm}int CSurface\-Child::Get\-X ()\hspace{0.3cm}{\tt  [virtual]}}\label{d3/d44/classCSurfaceChild_a4}


Gibt {\em x\/} zur\"{u}ck. 

\begin{Desc}
\item[R\"{u}ckgabe:]Gibt die x-Position des Objekts zur\"{u}ck. \end{Desc}
\begin{Desc}
\item[Siehe auch:]{\bf Get\-Y()}{\rm (S.\,\pageref{d3/d44/classCSurfaceChild_a5})}\end{Desc}


Definiert in Zeile 31 der Datei surfacechild.cc.

Wird benutzt von CAnimation::Draw().\index{CSurfaceChild@{CSurface\-Child}!GetY@{GetY}}
\index{GetY@{GetY}!CSurfaceChild@{CSurface\-Child}}
\subsubsection{\setlength{\rightskip}{0pt plus 5cm}int CSurface\-Child::Get\-Y ()\hspace{0.3cm}{\tt  [virtual]}}\label{d3/d44/classCSurfaceChild_a5}


Gibt {\em y\/} zur\"{u}ck. 

\begin{Desc}
\item[R\"{u}ckgabe:]Gibt die y-Position des Objekts zur\"{u}ck. \end{Desc}
\begin{Desc}
\item[Siehe auch:]{\bf Get\-X()}{\rm (S.\,\pageref{d3/d44/classCSurfaceChild_a4})}\end{Desc}


Definiert in Zeile 36 der Datei surfacechild.cc.

Wird benutzt von CAnimation::Draw().\index{CSurfaceChild@{CSurface\-Child}!SetDestinationSurface@{SetDestinationSurface}}
\index{SetDestinationSurface@{SetDestinationSurface}!CSurfaceChild@{CSurface\-Child}}
\subsubsection{\setlength{\rightskip}{0pt plus 5cm}void CSurface\-Child::Set\-Destination\-Surface (SDL\_\-Surface $\ast$ {\em dest})\hspace{0.3cm}{\tt  [virtual]}}\label{d3/d44/classCSurfaceChild_a2}


Setzt {\em destination\/} auf {\em dest\/}. 

Diese Funktion wird automatisch vom Constructor aufgerufen, und muss so, nur manuell aufgerufen werden, wenn {\em destination\/} nachtr\"{a}glich ge\"{a}ndert werden soll. \begin{Desc}
\item[Siehe auch:]{\bf CSurface\-Child()}{\rm (S.\,\pageref{d3/d44/classCSurfaceChild_a0})}\end{Desc}


Definiert in Zeile 20 der Datei surfacechild.cc.

Benutzt destination.

Wird benutzt von CSurface\-Child().\index{CSurfaceChild@{CSurface\-Child}!SetPosition@{SetPosition}}
\index{SetPosition@{SetPosition}!CSurfaceChild@{CSurface\-Child}}
\subsubsection{\setlength{\rightskip}{0pt plus 5cm}void CSurface\-Child::Set\-Position (int {\em newx}, int {\em newy})\hspace{0.3cm}{\tt  [virtual]}}\label{d3/d44/classCSurfaceChild_a3}


Gibt die Position an, an der das Bild gespeichert werden soll. 

Die angegebene Position ist relativ zu der Position von {\em destination\/}. D.h, wenn {\em destination\/} auf dem Bildschirm die Position (10/20) hat, und Set\-Position(5, 5) aufgerufen wird, wird auf dem Bildschirm (bei einem Aufruf von Draw()) an der Position (15/25) gezeichnet. \begin{Desc}
\item[Parameter:]
\begin{description}
\item[{\em newx}]ist die relative x-Position. \item[{\em newy}]ist die relative y-Position. \end{description}
\end{Desc}
\begin{Desc}
\item[Siehe auch:]{\bf Set\-Destination\-Surface()}{\rm (S.\,\pageref{d3/d44/classCSurfaceChild_a2})}\end{Desc}
\begin{Desc}
\item[Beispiele: ]\par
{\bf CAnimation1.cc}.\end{Desc}


Definiert in Zeile 25 der Datei surfacechild.cc.

Benutzt x und y.

Wird benutzt von CImage::Draw().

\subsection{Dokumentation der Datenelemente}
\index{CSurfaceChild@{CSurface\-Child}!destination@{destination}}
\index{destination@{destination}!CSurfaceChild@{CSurface\-Child}}
\subsubsection{\setlength{\rightskip}{0pt plus 5cm}SDL\_\-Surface$\ast$ {\bf CSurface\-Child::destination}}\label{d3/d44/classCSurfaceChild_o0}


Das SDL\_\-Surface in dem die Ausgabe stattfinden bzw. in dem gezeichnet wird. 



Definiert in Zeile 28 der Datei surfacechild.h.

Wird benutzt von Set\-Destination\-Surface().\index{CSurfaceChild@{CSurface\-Child}!x@{x}}
\index{x@{x}!CSurfaceChild@{CSurface\-Child}}
\subsubsection{\setlength{\rightskip}{0pt plus 5cm}int {\bf CSurface\-Child::x}}\label{d3/d44/classCSurfaceChild_o1}


Die x-Position des Objekts relativ zu {\em destination\/}. 



Definiert in Zeile 31 der Datei surfacechild.h.

Wird benutzt von Set\-Position().\index{CSurfaceChild@{CSurface\-Child}!y@{y}}
\index{y@{y}!CSurfaceChild@{CSurface\-Child}}
\subsubsection{\setlength{\rightskip}{0pt plus 5cm}int {\bf CSurface\-Child::y}}\label{d3/d44/classCSurfaceChild_o2}


Die y-Position des Objekts relativ zu {\em destination\/}. 



Definiert in Zeile 34 der Datei surfacechild.h.

Wird benutzt von Set\-Position().

Die Dokumentation f\"{u}r diese Klasse wurde erzeugt aufgrund der Dateien:\begin{CompactItemize}
\item 
src/{\bf surfacechild.h}\item 
src/{\bf surfacechild.cc}\end{CompactItemize}
