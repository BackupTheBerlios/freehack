\hypertarget{example__clist__1_8cc}{
\section{versuch2/example\_\-clist\_\-1.cc-Dateireferenz}
\label{d7/d91/example__clist__1_8cc}\index{versuch2/example_clist_1.cc@{versuch2/example\_\-clist\_\-1.cc}}
}
Diese Datei enth\"{a}lt ein Beispielprogramm f\"{u}r \hyperlink{classCList}{CList}. 

{\tt \#include $<$string$>$}\par
{\tt \#include $<$iostream$>$}\par
{\tt \#include \char`\"{}list.h\char`\"{}}\par
\subsection*{Funktionen}
\begin{CompactItemize}
\item 
int \hyperlink{example__clist__1_8cc_a0}{main} ()
\end{CompactItemize}


\subsection{Ausf\"{u}hrliche Beschreibung}
Diese Datei enth\"{a}lt ein Beispielprogramm f\"{u}r \hyperlink{classCList}{CList}. 

Es werden Eingaben entgegen genommen, die in der Liste gespeichert werden. Bei der Eingabe eines Punktes wird die Liste ausgegeben und das Programm beendet.

\begin{Desc}
\item[Autor:]Bodo Akdeniz \end{Desc}
\begin{Desc}
\item[Datum:]03.04.05 \end{Desc}
\begin{Desc}
\item[Version:]0.1\end{Desc}


Definiert in Datei \hyperlink{example__clist__1_8cc-source}{example\_\-clist\_\-1.cc}.

\subsection{Dokumentation der Funktionen}
\hypertarget{example__clist__1_8cc_a0}{
\index{example_clist_1.cc@{example\_\-clist\_\-1.cc}!main@{main}}
\index{main@{main}!example_clist_1.cc@{example\_\-clist\_\-1.cc}}
\subsubsection[main]{\setlength{\rightskip}{0pt plus 5cm}int main ()}}
\label{d7/d91/example__clist__1_8cc_a0}




Definiert in Zeile 18 der Datei example\_\-clist\_\-1.cc.

Benutzt CList$<$ T $>$::Add() und CList$<$ T $>$::Count().