\section{CList1.cc}
Ein Beispiel f\"{u}r die Benutzung von {\bf CList}{\rm (S.\,\pageref{da/d2a/classCList})}.



\footnotesize\begin{verbatim}/* Beispielprogramm f�r CList.
 *
 * Es werden Eingaben entgegen genommen, die in der Liste gespeichert werden.
 * Bei der Eingabe eines Punktes wird die Liste ausgegeben und das Programm
 * beendet.
 */

#include <string>
#include <iostream>
#include "../src/list.h"  // Enth�lt CList

int main()
{
        CList<std::string> liste;  // Ein Liste aus std::string-Elementen
        std::string temp;
        unsigned int i;

        while (1)
        {
                std::cout << "Eingabe: ";
                std::cin >> temp;
                if (temp == ".")
                        break;
                liste.Add(temp);
        }

        for (i=0; i<liste.Count(); i++)  // von 0 bis [Anzahl der Listenelemente]
        {
                std::cout << (i+1) << ": " << liste[i] << std::endl;
        }

        return 0;

}
\end{verbatim}
\normalsize
 