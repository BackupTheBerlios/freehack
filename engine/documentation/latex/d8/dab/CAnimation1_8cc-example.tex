\section{CAnimation1.cc}
Ein Beispiel f\"{u}r die Benutzung von {\bf CAnimation}{\rm (S.\,\pageref{da/d99/classCAnimation})}.



\footnotesize\begin{verbatim}/* Beispiel zur Benutzung von CAnimation.
 *
 * Es werden 2 Bilder geladen und der Animation hinzugef�gt, und die
 * Spielschleife wird gestartet.
 * jetzt kann durch dr�cken der Leertaste die Animation gestartet und gestoppt
 * werden. Beendet wird das Programm mit [ESC].
 * Es wird davon ausgegangen, dass der grunds�tzliche Umgang mit SDL
 * bekonnt ist.
 */

#include <SDL/SDL.h>
#include <iostream>

#include "../src/animation.h"           // f�r CAnimation
#include "../src/image.h"       // f�r CAnimationImage

int main()
{
        SDL_Surface *screen;
        int done=0;
        CAnimation *animation;          // Unsere Animation
        CImage *pic1, *pic2;    // Die 2 Bilder f�r die Animation
        
        // SDL stuff
        if (SDL_Init(SDL_INIT_VIDEO|SDL_INIT_NOPARACHUTE) < 0)
        {
                std::cout << "Unable to init SDL: " << SDL_GetError() << std::endl;
                return 1;
        }
        atexit(SDL_Quit);

        if ((screen=SDL_SetVideoMode(200, 200, 32, SDL_HWSURFACE|SDL_DOUBLEBUF)) == NULL)
        {
                std::cout << "Unable to set videomode: " << SDL_GetError() << std::endl;
                return 1;
        }
        
        
        // Bilder laden und DelayTime setzten
        
        /* Was hier als Ausgabe �bergeben wird ist egal, da es sowiso von
         * CAnimation �berschrieben wird, wenn man das Bild hinzuf�gt.
         * Man darf nur nicht  versuchen das Bild selbst zu zeichnen.
         */
        pic1 = new CImage(NULL);  
        pic1->LoadFromFile("../images/examples/zahnrad1.bmp");

        pic2 = new CImage(NULL);
        pic2->LoadFromFile("../images/examples/zahnrad2.bmp");

        
        // Animation vorbereiten
        animation = new CAnimation(screen);  // initialisierung; Ausgabe soll auf screen stattfinden
        animation->ImageList.Add(pic1);
        animation->ImageList.Add(pic2);
        animation->SetAnimationspeed(0.8);
        animation->SetPosition(0, 0);

        while (done == 0)  // Spielschleife
        {
                SDL_Event event;
                // Events abarbeiten
                while (SDL_PollEvent(&event))
                {
                        if (event.type == SDL_QUIT)
                                done = 1;
                        
                        if (event.type == SDL_KEYDOWN)
                        {
                                if (event.key.keysym.sym == SDLK_ESCAPE)
                                        done = 1;
                                if (event.key.keysym.sym == SDLK_SPACE)
                                {
                                        if (animation->IsRunning())
                                                animation->Stop();
                                        else
                                                animation->Start();
                                }
                        }
                }
        
                animation->Draw();  // aktuelles Bild der Animation zeichnen
                
                SDL_Flip(screen);  // Bild aktualisieren
        }
}
\end{verbatim}
\normalsize
 