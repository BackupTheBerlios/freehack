\section{CImage Klassenreferenz}
\label{d9/d8d/classCImage}\index{CImage@{CImage}}
Klasse zum laden/zeichnen von Bildern.  


{\tt \#include $<$image.h$>$}

Klassendiagramm f\"{u}r CImage::\begin{figure}[H]
\begin{center}
\leavevmode
\includegraphics[height=2cm]{d9/d8d/classCImage}
\end{center}
\end{figure}
\subsection*{\"{O}ffentliche Methoden}
\begin{CompactItemize}
\item 
{\bf CImage} (SDL\_\-Surface $\ast$dest)
\begin{CompactList}\small\item\em Constructor. \item\end{CompactList}\item 
virtual {\bf $\sim$CImage} ()
\begin{CompactList}\small\item\em Destructor. \item\end{CompactList}\item 
virtual void {\bf Load\-From\-File} (std::string file)
\begin{CompactList}\small\item\em L\"{a}d ein Bild. \item\end{CompactList}\item 
virtual void {\bf Set\-Color\-Key} (Uint8 r, Uint8 g, Uint8 b)
\begin{CompactList}\small\item\em Setzt die Farbe die transparent dargestellt werden soll. \item\end{CompactList}\item 
virtual void {\bf Set\-Alpha} (Uint8 a)
\begin{CompactList}\small\item\em Setzt den Grad der Transparenz. \item\end{CompactList}\item 
virtual void {\bf Set\-Image} (SDL\_\-Surface $\ast$img)
\begin{CompactList}\small\item\em Setzt {\em image\/}. \item\end{CompactList}\item 
virtual void {\bf Set\-Source\-Rect} (SDL\_\-Rect $\ast$src)
\begin{CompactList}\small\item\em Gibt ein Rechteck von {\em image\/} an, das gezeichnet werden soll. \item\end{CompactList}\item 
virtual void {\bf Set\-Source\-Rect} (int rx, int ry, int rw, int rh)
\begin{CompactList}\small\item\em Gibt ein Rechteck von {\em image\/} an, das gezeichnet werden soll. \item\end{CompactList}\item 
virtual void {\bf Draw} ()
\begin{CompactList}\small\item\em Zeichnet {\em image\/}. \item\end{CompactList}\item 
virtual void {\bf Draw} (int xpos, int ypos)
\begin{CompactList}\small\item\em Zeichnet {\em image\/} an die ({\em xpos/{\em ypos\/})\/}. \item\end{CompactList}\item 
virtual SDL\_\-Surface $\ast$ {\bf Get\-Surface} ()
\begin{CompactList}\small\item\em Gibt {\em image\/} zur\"{u}ck. \item\end{CompactList}\end{CompactItemize}
\subsection*{\"{O}ffentliche Attribute}
\begin{CompactItemize}
\item 
SDL\_\-Surface $\ast$ {\bf image}
\begin{CompactList}\small\item\em Zeigt auf das SDL\_\-Surface des Bildes. \item\end{CompactList}\item 
SDL\_\-Rect $\ast$ {\bf sourcerect}
\begin{CompactList}\small\item\em Definiert eine Rechteckige Fl\"{a}che von {\em image\/}, die gezeichnet werden soll. Ist {\em sourcerect\/} NULL wird, bei einem Aufruf von {\bf Draw()}{\rm (S.\,\pageref{d9/d8d/classCImage_a8})}, image komplett gezeichnet. \item\end{CompactList}\item 
std::string {\bf filename}
\begin{CompactList}\small\item\em Der Dateiname des geladenen Bildes. Kann auch leer sein, wenn {\em image\/} durch {\bf Set\-Image()}{\rm (S.\,\pageref{d9/d8d/classCImage_a5})} gesetzt wurde. \item\end{CompactList}\end{CompactItemize}


\subsection{Ausf\"{u}hrliche Beschreibung}
Klasse zum laden/zeichnen von Bildern. 

Eine Klasse um ein Bild (Bitmap) zu laden, speichern und zu zeichnen. Transparenz wird sowohl durch einen \char`\"{}Colorkey\char`\"{} als auch durch einen Alphawert unterst\"{u}tzt.

\begin{Desc}
\item[Autor:]Bodo Akdeniz \end{Desc}
\begin{Desc}
\item[Datum:]01.04.05 \end{Desc}
\begin{Desc}
\item[Version:]0.2 \end{Desc}
\begin{Desc}
\item[Beispiele: ]\par


{\bf CAnimation1.cc}.\end{Desc}




Definiert in Zeile 29 der Datei image.h.

\subsection{Beschreibung der Konstruktoren und Destruktoren}
\index{CImage@{CImage}!CImage@{CImage}}
\index{CImage@{CImage}!CImage@{CImage}}
\subsubsection{\setlength{\rightskip}{0pt plus 5cm}CImage::CImage (SDL\_\-Surface $\ast$ {\em dest})}\label{d9/d8d/classCImage_a0}


Constructor. 

Wird beim initiieren der Klasse aufgerufen und setzt das Surface auf dem gezeichnet werden soll.

\begin{Desc}
\item[Parameter:]
\begin{description}
\item[{\em dest}]ist ein Zeiger auf das SDL\_\-Surface, auf dem bei einem Aufruf von {\bf Draw()}{\rm (S.\,\pageref{d9/d8d/classCImage_a8})}, gezeichnet werden soll. \end{description}
\end{Desc}
\begin{Desc}
\item[Siehe auch:]{\bf Set\-Destination\-Surface()}{\rm (S.\,\pageref{d3/d44/classCSurfaceChild_a2})}\end{Desc}


Definiert in Zeile 11 der Datei image.cc.

Benutzt Set\-Source\-Rect().\index{CImage@{CImage}!~CImage@{$\sim$CImage}}
\index{~CImage@{$\sim$CImage}!CImage@{CImage}}
\subsubsection{\setlength{\rightskip}{0pt plus 5cm}CImage::$\sim${\bf CImage} ()\hspace{0.3cm}{\tt  [virtual]}}\label{d9/d8d/classCImage_a1}


Destructor. 



Definiert in Zeile 16 der Datei image.cc.

\subsection{Dokumentation der Elementfunktionen}
\index{CImage@{CImage}!Draw@{Draw}}
\index{Draw@{Draw}!CImage@{CImage}}
\subsubsection{\setlength{\rightskip}{0pt plus 5cm}void CImage::Draw (int {\em xpos}, int {\em ypos})\hspace{0.3cm}{\tt  [virtual]}}\label{d9/d8d/classCImage_a9}


Zeichnet {\em image\/} an die ({\em xpos/{\em ypos\/})\/}. 

Genauso wie {\bf Draw()}{\rm (S.\,\pageref{d9/d8d/classCImage_a8})}, nur dass davor Set\-Position(xpos, ypos) aufgerufen wird. \begin{Desc}
\item[Siehe auch:]{\bf Draw()}{\rm (S.\,\pageref{d9/d8d/classCImage_a8})} 

Set\-Positino()\end{Desc}


Definiert in Zeile 67 der Datei image.cc.

Benutzt Draw() und CSurface\-Child::Set\-Position().\index{CImage@{CImage}!Draw@{Draw}}
\index{Draw@{Draw}!CImage@{CImage}}
\subsubsection{\setlength{\rightskip}{0pt plus 5cm}void CImage::Draw ()\hspace{0.3cm}{\tt  [virtual]}}\label{d9/d8d/classCImage_a8}


Zeichnet {\em image\/}. 

Die Funktion zeichnet den, mit {\bf Set\-Source\-Rect()}{\rm (S.\,\pageref{d9/d8d/classCImage_a6})} angegebenen Teil, von {\em image\/} an die mit {\bf Set\-Position()}{\rm (S.\,\pageref{d3/d44/classCSurfaceChild_a3})} festgelegte Position in das SDL\_\-Surface, das beim initialisieren mit {\bf CImage()}{\rm (S.\,\pageref{d9/d8d/classCImage_a0})} oder manuell mit {\bf Set\-Destination\-Surface()}{\rm (S.\,\pageref{d3/d44/classCSurfaceChild_a2})}, angegeben wurde. \begin{Desc}
\item[Siehe auch:]{\bf Set\-Source\-Rect()}{\rm (S.\,\pageref{d9/d8d/classCImage_a6})} 

{\bf Set\-Position}{\rm (S.\,\pageref{d3/d44/classCSurfaceChild_a3})} 

{\bf CImage()}{\rm (S.\,\pageref{d9/d8d/classCImage_a0})} 

{\bf Set\-Destination\-Surface()}{\rm (S.\,\pageref{d3/d44/classCSurfaceChild_a2})} 

{\bf Draw(int, int)}{\rm (S.\,\pageref{d9/d8d/classCImage_a9})}\end{Desc}


Definiert in Zeile 59 der Datei image.cc.

Benutzt image und sourcerect.

Wird benutzt von Draw().\index{CImage@{CImage}!GetSurface@{GetSurface}}
\index{GetSurface@{GetSurface}!CImage@{CImage}}
\subsubsection{\setlength{\rightskip}{0pt plus 5cm}SDL\_\-Surface $\ast$ CImage::Get\-Surface ()\hspace{0.3cm}{\tt  [virtual]}}\label{d9/d8d/classCImage_a10}


Gibt {\em image\/} zur\"{u}ck. 

Gibt einen Zeiger auf das SDL\_\-Surface zur\"{u}ck, das das zu zeichnende Bild enth\"{a}lt. \begin{Desc}
\item[R\"{u}ckgabe:]Zeiger auf das SDL\_\-Surface, das das Bild enth\"{a}lt ({\em image\/}).\end{Desc}


Definiert in Zeile 73 der Datei image.cc.\index{CImage@{CImage}!LoadFromFile@{LoadFromFile}}
\index{LoadFromFile@{LoadFromFile}!CImage@{CImage}}
\subsubsection{\setlength{\rightskip}{0pt plus 5cm}void CImage::Load\-From\-File (std::string {\em file})\hspace{0.3cm}{\tt  [virtual]}}\label{d9/d8d/classCImage_a2}


L\"{a}d ein Bild. 

{\bf Load\-From\-File()}{\rm (S.\,\pageref{d9/d8d/classCImage_a2})} l\"{a}d ein Bitmap in image, das dann verwendet werden kann. Tritt ein Fehler auf ist {\em image\/} NULL und {\em filename\/} leer. \begin{Desc}
\item[Siehe auch:]{\bf Set\-Image()}{\rm (S.\,\pageref{d9/d8d/classCImage_a5})}\end{Desc}
\begin{Desc}
\item[Beispiele: ]\par
{\bf CAnimation1.cc}.\end{Desc}


Definiert in Zeile 20 der Datei image.cc.

Benutzt filename und image.\index{CImage@{CImage}!SetAlpha@{SetAlpha}}
\index{SetAlpha@{SetAlpha}!CImage@{CImage}}
\subsubsection{\setlength{\rightskip}{0pt plus 5cm}void CImage::Set\-Alpha (Uint8 {\em a})\hspace{0.3cm}{\tt  [virtual]}}\label{d9/d8d/classCImage_a4}


Setzt den Grad der Transparenz. 

\begin{Desc}
\item[Parameter:]
\begin{description}
\item[{\em a}]bestimmt den Grad der Transparenz. (0=keine Transparenz bis 255=unsichtbar) \end{description}
\end{Desc}
\begin{Desc}
\item[Siehe auch:]{\bf Set\-Color\-Key()}{\rm (S.\,\pageref{d9/d8d/classCImage_a3})}\end{Desc}


Definiert in Zeile 34 der Datei image.cc.

Benutzt image.\index{CImage@{CImage}!SetColorKey@{SetColorKey}}
\index{SetColorKey@{SetColorKey}!CImage@{CImage}}
\subsubsection{\setlength{\rightskip}{0pt plus 5cm}void CImage::Set\-Color\-Key (Uint8 {\em r}, Uint8 {\em g}, Uint8 {\em b})\hspace{0.3cm}{\tt  [virtual]}}\label{d9/d8d/classCImage_a3}


Setzt die Farbe die transparent dargestellt werden soll. 

Setzt die Farbe, die beim zeichnen mit {\bf Draw()}{\rm (S.\,\pageref{d9/d8d/classCImage_a8})} nicht gezeichnet werden soll, also transparent dargestellt wird. \begin{Desc}
\item[Parameter:]
\begin{description}
\item[{\em r}]bestimmt den Rotanteil der Farbe. (0=kein Rot bis 255=Sehr Rot) \item[{\em g}]bestimmt den Gr\"{u}nanteil der Farbe. (0=kein Gr\"{u}n bis 255=Sehr Gr\"{u}n) \item[{\em b}]bestimmt den Blauanteil der Farbe. (0=kein Blau bis 255=Sehr Blau) \end{description}
\end{Desc}
\begin{Desc}
\item[Siehe auch:]{\bf Set\-Alpha()}{\rm (S.\,\pageref{d9/d8d/classCImage_a4})}\end{Desc}


Definiert in Zeile 29 der Datei image.cc.

Benutzt image.\index{CImage@{CImage}!SetImage@{SetImage}}
\index{SetImage@{SetImage}!CImage@{CImage}}
\subsubsection{\setlength{\rightskip}{0pt plus 5cm}void CImage::Set\-Image (SDL\_\-Surface $\ast$ {\em img})\hspace{0.3cm}{\tt  [virtual]}}\label{d9/d8d/classCImage_a5}


Setzt {\em image\/}. 

Die Funktion kann verwendet werden, wenn direkt ein SDL\_\-Surface \"{u}bergeben werden und kein Bild aus einer Datei geladen werden soll. Um ein Bild aus einer Datei zu laden verwenden sie bitte {\bf Load\-From\-File()}{\rm (S.\,\pageref{d9/d8d/classCImage_a2})}. \begin{Desc}
\item[Parameter:]
\begin{description}
\item[{\em img}]ist ein Zeiger auf das SDL\_\-Surface, das als {\em image\/} verwendet werden soll. \end{description}
\end{Desc}
\begin{Desc}
\item[Siehe auch:]{\bf Load\-From\-File()}{\rm (S.\,\pageref{d9/d8d/classCImage_a2})}\end{Desc}


Definiert in Zeile 39 der Datei image.cc.

Benutzt image.\index{CImage@{CImage}!SetSourceRect@{SetSourceRect}}
\index{SetSourceRect@{SetSourceRect}!CImage@{CImage}}
\subsubsection{\setlength{\rightskip}{0pt plus 5cm}void CImage::Set\-Source\-Rect (int {\em rx}, int {\em ry}, int {\em rw}, int {\em rh})\hspace{0.3cm}{\tt  [virtual]}}\label{d9/d8d/classCImage_a7}


Gibt ein Rechteck von {\em image\/} an, das gezeichnet werden soll. 

Die Funktion hat die selbe Funktionalit\"{a}t wie {\bf Set\-Source\-Rect(SDL\_\-Rect $\ast$)}{\rm (S.\,\pageref{d9/d8d/classCImage_a6})}, nur kann man hier die Koordinaten, L\"{a}nge und Breite direkt \"{u}bergeben. \begin{Desc}
\item[Parameter:]
\begin{description}
\item[{\em rx}]ist die x-Koordinate des Rechtecks innerhalb von {\em image\/}. \item[{\em ry}]ist die y-Koordinate des Rechtecks innerhalb von {\em image\/}. \item[{\em rw}]ist die Breite des Rechtecks. \item[{\em rh}]ist die H\"{o}he des Rechtecks. \end{description}
\end{Desc}
\begin{Desc}
\item[Siehe auch:]{\bf Set\-Source\-Rect(SDL\_\-Rect $\ast$)}{\rm (S.\,\pageref{d9/d8d/classCImage_a6})}\end{Desc}


Definiert in Zeile 49 der Datei image.cc.

Benutzt Set\-Source\-Rect().\index{CImage@{CImage}!SetSourceRect@{SetSourceRect}}
\index{SetSourceRect@{SetSourceRect}!CImage@{CImage}}
\subsubsection{\setlength{\rightskip}{0pt plus 5cm}void CImage::Set\-Source\-Rect (SDL\_\-Rect $\ast$ {\em src})\hspace{0.3cm}{\tt  [virtual]}}\label{d9/d8d/classCImage_a6}


Gibt ein Rechteck von {\em image\/} an, das gezeichnet werden soll. 

Diese Funktion kann verwendet werden, wenn nur ein Teil, von {\em image\/} gezeichnet werden soll. Soll {\em image\/} komplett gezeichnet werden, kann f\"{u}r {\em src\/} NULL \"{u}bergeben werden, das ist aber sofern es nicht manuell ge\"{a}ndert wurde, sowiso so. \begin{Desc}
\item[Parameter:]
\begin{description}
\item[{\em src}]ist ein Zeiger auf ein SDL\_\-Rect, und bestimmt ein Rechteck von {\em image\/}. \end{description}
\end{Desc}
\begin{Desc}
\item[Siehe auch:]{\bf Set\-Source\-Rect(int, int, int, int)}{\rm (S.\,\pageref{d9/d8d/classCImage_a7})}\end{Desc}


Definiert in Zeile 44 der Datei image.cc.

Benutzt sourcerect.

Wird benutzt von CImage() und Set\-Source\-Rect().

\subsection{Dokumentation der Datenelemente}
\index{CImage@{CImage}!filename@{filename}}
\index{filename@{filename}!CImage@{CImage}}
\subsubsection{\setlength{\rightskip}{0pt plus 5cm}std::string {\bf CImage::filename}}\label{d9/d8d/classCImage_o2}


Der Dateiname des geladenen Bildes. Kann auch leer sein, wenn {\em image\/} durch {\bf Set\-Image()}{\rm (S.\,\pageref{d9/d8d/classCImage_a5})} gesetzt wurde. 



Definiert in Zeile 39 der Datei image.h.

Wird benutzt von Load\-From\-File().\index{CImage@{CImage}!image@{image}}
\index{image@{image}!CImage@{CImage}}
\subsubsection{\setlength{\rightskip}{0pt plus 5cm}SDL\_\-Surface$\ast$ {\bf CImage::image}}\label{d9/d8d/classCImage_o0}


Zeigt auf das SDL\_\-Surface des Bildes. 



Definiert in Zeile 35 der Datei image.h.

Wird benutzt von Draw(), Load\-From\-File(), Set\-Alpha(), Set\-Color\-Key() und Set\-Image().\index{CImage@{CImage}!sourcerect@{sourcerect}}
\index{sourcerect@{sourcerect}!CImage@{CImage}}
\subsubsection{\setlength{\rightskip}{0pt plus 5cm}SDL\_\-Rect$\ast$ {\bf CImage::sourcerect}}\label{d9/d8d/classCImage_o1}


Definiert eine Rechteckige Fl\"{a}che von {\em image\/}, die gezeichnet werden soll. Ist {\em sourcerect\/} NULL wird, bei einem Aufruf von {\bf Draw()}{\rm (S.\,\pageref{d9/d8d/classCImage_a8})}, image komplett gezeichnet. 



Definiert in Zeile 37 der Datei image.h.

Wird benutzt von Draw() und Set\-Source\-Rect().

Die Dokumentation f\"{u}r diese Klasse wurde erzeugt aufgrund der Dateien:\begin{CompactItemize}
\item 
src/{\bf image.h}\item 
src/{\bf image.cc}\end{CompactItemize}
