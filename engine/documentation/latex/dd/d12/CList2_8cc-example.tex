\section{CList2.cc}
Ein Beispiel f\"{u}r die Benutzung von {\bf CList}{\rm (S.\,\pageref{da/d2a/classCList})}.



\footnotesize\begin{verbatim}/* Ein weiteres Beispiel f�r CList.
 * 
 * Hier werden wie bei dem ersten Beispiel Eingaben entgegengenommen,
 * und in der Liste gespeichert. Wenn "." eingegeben wird wird die Eingabe-
 * schleife beendet und die Liste wird ausgegeben.
 * Der Unterschied zu dem ersten Beispielprogramm besteht darin, dass
 * die Liste aus Zeigern auf eine Klasse besteht.
 * In dem Fall kann man auch NULL zur Liste hinzuf�gen, und den das
 * Objekt erst sp�ter anlegen, was in diesem Beispiel gezeigt wird.
 */

#include <string>
#include <iostream>
#include "../src/list.h"

class contact {
public:
        std::string name;
        std::string phonenumber;

        contact() {
        }

        contact(std::string n, std::string p) {
                name = n;
                phonenumber = p;
        }
};

int main()
{
        CList<contact*> contacts;
        std::string name, phone;
        unsigned int i;

        while (1)
        {
                std::cout << "Name: ";
                std::cin >> name;

                if (name == ".")
                        break;
                
                std::cout << "Phone: ";
                std::cin >> phone;
                contacts.Add(NULL);
                contacts[contacts.Count()-1] = new contact(name, phone);
        }

        for (i=0; i<contacts.Count(); i++)  // von 0 bis [Anzahl der Listenelemente]
        {
                std::cout << (i+1) << ". Name: " << contacts[i]->name << std::endl;
                std::cout << (i+1) << ". Phone: " << contacts[i]->phonenumber << std::endl << std::endl;;
        }
}
\end{verbatim}
\normalsize
 