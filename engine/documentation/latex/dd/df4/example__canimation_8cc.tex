\hypertarget{example__canimation_8cc}{
\section{versuch2/example\_\-canimation.cc-Dateireferenz}
\label{dd/df4/example__canimation_8cc}\index{versuch2/example_canimation.cc@{versuch2/example\_\-canimation.cc}}
}
In dieser Datei ist ein kleines Beispiel zur Benutzung von \hyperlink{classCAnimation}{CAnimation}. 

{\tt \#include $<$SDL/SDL.h$>$}\par
{\tt \#include $<$iostream$>$}\par
{\tt \#include \char`\"{}animation.h\char`\"{}}\par
{\tt \#include \char`\"{}animationimage.h\char`\"{}}\par
\subsection*{Funktionen}
\begin{CompactItemize}
\item 
int \hyperlink{example__canimation_8cc_a0}{main} ()
\end{CompactItemize}


\subsection{Ausf\"{u}hrliche Beschreibung}
In dieser Datei ist ein kleines Beispiel zur Benutzung von \hyperlink{classCAnimation}{CAnimation}. 

Es werden 2 Bilder geladen und der Animation hinzugef\"{u}gt, und die Spielschleife wird gestartet. jetzt kann durch dr\"{u}cken der Leertaste die Animation gestartet und gestoppt werden. Beendet wird das Programm mit \mbox{[}ESC\mbox{]}. Es wird davon ausgegangen, dass der grunds\"{a}tzliche Umgang mit SDL bekonnt ist.

\begin{Desc}
\item[Autor:]Bodo Akdeniz \end{Desc}
\begin{Desc}
\item[Datum:]02.04.05 \end{Desc}
\begin{Desc}
\item[Version:]0.2\end{Desc}


Definiert in Datei \hyperlink{example__canimation_8cc-source}{example\_\-canimation.cc}.

\subsection{Dokumentation der Funktionen}
\hypertarget{example__canimation_8cc_a0}{
\index{example_canimation.cc@{example\_\-canimation.cc}!main@{main}}
\index{main@{main}!example_canimation.cc@{example\_\-canimation.cc}}
\subsubsection[main]{\setlength{\rightskip}{0pt plus 5cm}int main ()}}
\label{dd/df4/example__canimation_8cc_a0}




Definiert in Zeile 24 der Datei example\_\-canimation.cc.

Benutzt CList$<$ T $>$::Add(), CAnimation::Image\-List, CAnimation::Set\-Animationspeed(), CSurface\-Child::Set\-Position(), CAnimation::Start\-Animation() und CAnimation::Stop\-Animation().